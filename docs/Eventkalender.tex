\RequirePackage[l2tabu, orthodox]{nag}
\documentclass[11pt,a4paper,titlepage,portrait,ngerman]{scrartcl}
\pdfgentounicode=1
\pdfcompresslevel=9

% INFO SECTION %
\def \docTitle{Eventkalender}
\def \docSubtitle{M151}
\def \docAuthor{Kim Jeker}
\def \docDate{\today}

% PACKAGES
%\usepackage[osf]{mathpazo} 
\usepackage[default]{raleway}% Font
\usepackage[pdftex,
	plainpages=true,
	breaklinks=true,
	colorlinks=false,
	linktoc=all,
	linkcolor=blue,
	unicode=true,
	pdfborder={0 0 0},
	pdftitle=\docTitle,
	pdfauthor=\docAuthor 
]{hyperref}
\usepackage[pdftex]{graphicx}
\usepackage{hfoldsty}
\usepackage{thumbpdf}
\usepackage{amsmath}
\usepackage[ngerman]{babel}
\usepackage{siunitx}
\usepackage{cleveref}
\usepackage{fancyhdr}
\usepackage[utf8]{inputenc}
\usepackage[T1]{fontenc}
\usepackage{ngerman}
\usepackage[a4paper]{geometry}
\usepackage[
	letterspace=100,
	babel=true,
	tracking=true,
	kerning=true,
	protrusion=true,
	expansion
]{microtype}
\usepackage{fixltx2e}
\usepackage{currfile}
\usepackage{caption}
\usepackage{xcolor}
\usepackage{lastpage}
\usepackage{ellipsis}
\usepackage{bigfoot}
\usepackage{lipsum}



\setkomafont{disposition}{\normalfont}
\setkomafont{title}{\normalfont\bfseries\scshape\Huge}
\setkomafont{subtitle}{\normalfont\scshape\bfseries\normalsize}
\setkomafont{dictumauthor}{\normalfont\bfseries\small}
\setkomafont{section}{\normalfont\bfseries\scshape\Large}
\setkomafont{subsection}{\normalfont\bfseries\scshape\large}
\setkomafont{subsubsection}{\normalfont\bfseries\normalsize}
\setkomafont{descriptionlabel}{\normalfont\bfseries\small}



% DOC INFO 
\title{\docTitle}
\subtitle{\docSubtitle}
\author{\large\scshape\docAuthor}
\date{\normalsize\scshape\docDate}

% HEADER & FOOTER
\pagestyle{fancy}
\lhead{\docTitle}
\chead{\textbf{\leftmark}}
\rhead{\docSubtitle}
\lfoot{\currfilename}
\cfoot{\textbf{\docAuthor}}
\rfoot{Seite \thepage\ von \pageref*{LastPage} }


% Make German ;)
\renewcommand{\figurename}{Abbildung}


\begin{document}
\maketitle
\newpage
\tableofcontents
\newpage
\section{Pflichtenheft}
\subsection{Funktionsbeschreibung}
Die Eventkalender-Applikation wird folgende Funktionen haben:
\begin{list}{\textendash}{\vspace{-5mm}}
	\item{\subsubsection*{Administartionsbereich}
		\begin{list}{\textbullet}{\vspace{-3mm}}
			\item{Neue Veranstaltung erstellen / editieren / löschen}
			\item{Neue Preisgruppen erstellen / editieren / löschen}
			\item{Neues Genre erstellen / editieren / löschen}
		\end{list}
	}
	\item{\vspace{-5mm}\subsubsection*{Frontend} 
		\begin{list}{\textbullet}{\vspace{-3mm}}
			\item{Anzeige des Programms in chronologischer Reihenfolge}
			\item{Archiv mit verganenen Veranstaltungen}
			\item{Filterung nach Genere und Datum}
		\end{list}
	}
\end{list}

\subsection{Darstellung}
\subsubsection*{Programm}
Das Proramm wird als vereinfachte Auflistung der Events in chronologischer Rheienfolge dargestellt. Über der Liste wird jeweils ein Datum (z. B. \textbf{25.10.2014}) ausgegeben, darunter die Events, welche an diesem Datum stattfinen. Events mit mehreren Terminen werden an den entspechenden Daten erneut aufgeführt.
\\

\noindent
Folgende Informationen soll die Auflistung beinhalten:

\begin{list}{\textendash}{}
	\item{Event-Name}
	\item{Zeit}
	\item{Event-Thumbnail}
\end{list}

\noindent
Bei Klick auf einen Eintrag wird das Detail mit allen weiteren Informationen angezeigt.
\\

\noindent
Zusätzlich befindet sich auf der Seite noch die Einstellung für den Genre- und Datumsfilter.

\subsubsection*{Archiv}
Die Darstellung der Events im Archiv ist die Selbe wie die in der Programm-Ansicht. Zusätzlich gibt es noch eine Seiten-Navigation, um zwischen den Seiten zu blättern.

Es weren alle Events im Archiv angezeigt, bei welchen keiner der Veranstaltungstermine in der Zunkunf liegt. 

Events, welche im Archiv sind, können immer noch bearbeitet werden, da man ja im Zweifelsfall auch nachträglich noch Fehler (z. B. Tippfehler) korrigieren möchte.


\subsection{Umsetzung}
\subsubsection*{Eingabevalidierung}
Die Eingabevalidierung von Formularfeldern wird sowohl Client- als auch Serverseitig realisiert. 
Auf Client-Seite wird dies, wo möglich, über die \textbf{HTML5-Input-Types} gelöst. Auf Serverseite wird überprüft, ob Pflichtfeler nicht leer sind und ob die ausgewählte Preisgruppe/Genre existiert.
Da Links theoretisch auch Anker-/E-Mail-Links sein können, werden diese nicht speziell gecheckt.

\subsubsection*{Genre \& Preisgruppe löschen}
Meine Idee ist, dass es immer mindestens eine Preisgruppe und ein Genre geben muss.
Der Administartor kann dann einstellen, welche Preisgruppe/welches Genre das «\textbf{Default-Genre}» bzw. die «\textbf{Default-Preisgruppe}» ist. Dies ist beim Anlegen eines neuen Events schon vorausgewählt. 
\\

\noindent
Wird nun eine Preisgruppe/ein Genre gelöscht, so wird, falls es Events gibt, welche der Preisgruppe/Genre zugewiesen sind, zuerst eine Meldung ausgegeben, welche den Benutzer warnt, und bei bestätigung, wird die Preisgruppe/Genre gelöscht und bei allen zugewiesenen Events die Default-Preisgruppe/Genre gesetzt.
\\

\noindent
Die Default-Preisgruppe/Genre kann nicht gelöscht werden.
\subsection{Technisches}
\subsubsection*{Frameworks}
Folgende Frameworks werden verwendet:
\begin{list}{\textendash}{}
\item{Laravel (PHP)}
\item{MooTools (JS)}
\item{Kube Web Framework (CSS/JS)}
\item{Font-Awesome (CSS/Glyphicons)}
\end{list}

\newpage
\section{Blackbox-Testplan}
\subsection{Erstellen einer Veranstaltung}
\subsubsection*{Name eingeben}
\vspace{2.5mm}
\begin{tabular}{|p{0.3\textwidth}||p{0.63\textwidth}|}
\hline \rule[-2ex]{0pt}{5.5ex} \textbf{Vorbedingung} & {
	\begin{list}{\textendash}{\vspace{-5mm}}
		\item{User ist angemeldet}
		\item{Berabeiten/Erstellen-Modus ist geöffnet}
	\end{list}
} \\ 
\hline \rule[-2ex]{0pt}{5.5ex} \textbf{Aktion} & {
	\begin{list}{\textendash}{\vspace{-5mm}}
		\item{Der User gibt den Namen in das vorgesehene Feld ein.}
	\end{list}
} \\ 
\hline \rule[-2ex]{0pt}{5.5ex} \textbf{Nachbedingung} & {
	\begin{list}{\textendash}{\vspace{-5mm}}
		\item{Der Event-Name steht im vorgesehenen Feld.}
	\end{list}
} \\ 
\hline 
\end{tabular}

\subsubsection*{Besetzung angeben}
\vspace{2.5mm}
\noindent
\begin{tabular}{|p{0.3\textwidth}||p{0.63\textwidth}|}
\hline \rule[-2ex]{0pt}{5.5ex} \textbf{Vorbedingung} & {
	\begin{list}{\textendash}{\vspace{-5mm}}
		\item{User ist angemeldet}
		\item{Berabeiten/Erstellen-Modus ist geöffnet}
	\end{list}
} \\ 
\hline \rule[-2ex]{0pt}{5.5ex} \textbf{Aktion} & {
	\begin{list}{\textendash}{\vspace{-5mm}}
		\item{Der User gibt die beteiligten Personen in das vorgesehene Feld ein.}
	\end{list}
} \\ 
\hline \rule[-2ex]{0pt}{5.5ex} \textbf{Nachbedingung} & {
	\begin{list}{\textendash}{\vspace{-5mm}}
		\item{Die beteiligten Personen steht im vorgesehenen Feld.}
	\end{list}
}  \\
\hline 
\end{tabular}

\subsubsection*{Beschreibung eingeben}
\vspace{2.5mm}
\noindent
\begin{tabular}{|p{0.3\textwidth}||p{0.63\textwidth}|}
\hline \rule[-2ex]{0pt}{5.5ex} \textbf{Vorbedingung} & {
	\begin{list}{\textendash}{\vspace{-5mm}}
		\item{User ist angemeldet}
		\item{Berabeiten/Erstellen-Modus ist geöffnet}
	\end{list}
} \\ 
\hline \rule[-2ex]{0pt}{5.5ex} \textbf{Aktion} & {
	\begin{list}{\textendash}{\vspace{-5mm}}
		\item{Der User gibt die Beschreibung in das vorgesehene Feld ein.}
	\end{list}
} \\ 
\hline \rule[-2ex]{0pt}{5.5ex} \textbf{Nachbedingung} & {
	\begin{list}{\textendash}{\vspace{-5mm}}
		\item{Die Beschreibung steht im vorgesehenen Feld.}
	\end{list}
}  \\
\hline 
\end{tabular}

\subsubsection*{Vorstellungsdaten erfassen}
\vspace{2.5mm}
\noindent
\begin{tabular}{|p{0.3\textwidth}||p{0.63\textwidth}|}
\hline \rule[-2ex]{0pt}{5.5ex} \textbf{Vorbedingung} & {
	\begin{list}{\textendash}{\vspace{-5mm}}
		\item{User ist angemeldet}
		\item{Berabeiten/Erstellen-Modus ist geöffnet}
	\end{list}
} \\ 
\hline \rule[-2ex]{0pt}{5.5ex} \textbf{Aktion} & {
	\begin{list}{\textendash}{\vspace{-5mm}}
		\item{Der User gibt einen Vorstellungstermin in das vorgesehene Feld ein.}
	\end{list}
} \\ 
\hline \rule[-2ex]{0pt}{5.5ex} \textbf{Nachbedingung} & {
	\begin{list}{\textendash}{\vspace{-5mm}}
		\item{Der Termin steht im vorgesehenen Feld. Der User sieht die Möglichkeit, einen weiteren Termin hinzuzufügen.}
	\end{list}
}  \\
\hline \rule[-2ex]{0pt}{5.5ex} \textbf{Aktion} & {
	\begin{list}{\textendash}{\vspace{-5mm}}
		\item{Der User gibt einen weiteren Vorstellungstermin in das vorgesehene Feld ein.}
	\end{list}
} \\ 
\hline \rule[-2ex]{0pt}{5.5ex} \textbf{Nachbedingung} & {
	\begin{list}{\textendash}{\vspace{-5mm}}
		\item{Der zweite Termin steht im vorgesehenen Feld. Der User sieht die Möglichkeit, einen weiteren Termin hinzuzufügen.}
	\end{list}
}  \\
\hline 
\multicolumn{2}{|c|}{{\large \textbf{usw}}\textellipsis} \\
\hline 
\end{tabular}

\subsubsection*{Vorstellungszeiten erfassen}
\vspace{2.5mm}
\noindent
\begin{tabular}{|p{0.3\textwidth}||p{0.63\textwidth}|}
\hline \rule[-2ex]{0pt}{5.5ex} \textbf{Vorbedingung} & {
	\begin{list}{\textendash}{\vspace{-5mm}}
		\item{User ist angemeldet}
		\item{Berabeiten/Erstellen-Modus ist geöffnet}
		\item{Mindestens ein Vorstellungstermin ist angelegt}
	\end{list}
} \\ 
\hline \rule[-2ex]{0pt}{5.5ex} \textbf{Aktion} & {
	\begin{list}{\textendash}{\vspace{-5mm}}
		\item{Der User gibt beim entsprechenden Vorstellungstermin die Veranstaltungszeit in das vorgesehene Feld ein.}
	\end{list}
} \\ 
\hline \rule[-2ex]{0pt}{5.5ex} \textbf{Nachbedingung} & {
	\begin{list}{\textendash}{\vspace{-5mm}}
		\item{Die Zeit steht im vorgesehenen Feld.}
	\end{list}
}  \\
\hline 
\multicolumn{2}{|c|}{{\textbf{Schritt für jeden Termin wiederholen}\textellipsis}} \\
\hline 
\end{tabular}

\subsubsection*{Dauer erfassen}
\vspace{2.5mm}
\noindent
\begin{tabular}{|p{0.3\textwidth}||p{0.63\textwidth}|}
\hline \rule[-2ex]{0pt}{5.5ex} \textbf{Vorbedingung} & {
	\begin{list}{\textendash}{\vspace{-5mm}}
		\item{User ist angemeldet}
		\item{Berabeiten/Erstellen-Modus ist geöffnet}
	\end{list}
} \\ 
\hline \rule[-2ex]{0pt}{5.5ex} \textbf{Aktion} & {
	\begin{list}{\textendash}{\vspace{-5mm}}
		\item{Der User gibt die Dauer in das vorgesehene Feld ein.}
	\end{list}
} \\ 
\hline \rule[-2ex]{0pt}{5.5ex} \textbf{Nachbedingung} & {
	\begin{list}{\textendash}{\vspace{-5mm}}
		\item{Die Dauer steht im vorgesehenen Feld.}
	\end{list}
}  \\
\hline 
\end{tabular}

\subsubsection*{Links erfassen}
\vspace{2.5mm}
\noindent
\begin{tabular}{|p{0.3\textwidth}||p{0.63\textwidth}|}
\hline \rule[-2ex]{0pt}{5.5ex} \textbf{Vorbedingung} & {
	\begin{list}{\textendash}{\vspace{-5mm}}
		\item{User ist angemeldet}
		\item{Berabeiten/Erstellen-Modus ist geöffnet}
	\end{list}
} \\ 
\hline \rule[-2ex]{0pt}{5.5ex} \textbf{Aktion} & {
	\begin{list}{\textendash}{\vspace{-5mm}}
		\item{Der User gibt den Link-Namen und die URL in das vorgesehene Feld ein.}
	\end{list}
} \\ 
\hline \rule[-2ex]{0pt}{5.5ex} \textbf{Nachbedingung} & {
	\begin{list}{\textendash}{\vspace{-5mm}}
		\item{Der Link-Name sowie die URL stehten in den vorgesehenen Felder. Der User sieht die Möglichkeit, einen weiteren Link hinzuzufügen.}
	\end{list}
}  \\
\hline \rule[-2ex]{0pt}{5.5ex} \textbf{Aktion} & {
	\begin{list}{\textendash}{\vspace{-5mm}}
		\item{Der User erfasst einen weiteren Link.}
	\end{list}
} \\ 
\hline \rule[-2ex]{0pt}{5.5ex} \textbf{Nachbedingung} & {
	\begin{list}{\textendash}{\vspace{-5mm}}
		\item{Der Link-Name sowie die URL des zweiten Links stehten in den vorgesehenen Felder. Der User sieht die Möglichkeit, einen weiteren Link hinzuzufügen.}
	\end{list}
}  \\
\hline 
\multicolumn{2}{|c|}{{\large \textbf{usw}}\textellipsis} \\
\hline 
\end{tabular}

\subsubsection*{Eventbild auswählen}
\vspace{2.5mm}
\noindent
\begin{tabular}{|p{0.3\textwidth}||p{0.63\textwidth}|}
\hline \rule[-2ex]{0pt}{5.5ex} \textbf{Vorbedingung} & {
	\begin{list}{\textendash}{\vspace{-5mm}}
		\item{User ist angemeldet}
		\item{Berabeiten/Erstellen-Modus ist geöffnet}
	\end{list}
} \\ 
\hline \rule[-2ex]{0pt}{5.5ex} \textbf{Aktion} & {
	\begin{list}{\textendash}{\vspace{-5mm}}
		\item{Der User wählt ein Bild (welches sich auf seinem Computer befindet) aus.}
	\end{list}
} \\ 
\hline \rule[-2ex]{0pt}{5.5ex} \textbf{Nachbedingung} & {
	\begin{list}{\textendash}{\vspace{-5mm}}
		\item{Ein Eventbild ist ausgewählt.}
	\end{list}
}  \\
\hline 
\end{tabular}

\subsubsection*{Bilbeschreibung einfügen}
\vspace{2.5mm}
\noindent
\begin{tabular}{|p{0.3\textwidth}||p{0.63\textwidth}|}
\hline \rule[-2ex]{0pt}{5.5ex} \textbf{Vorbedingung} & {
	\begin{list}{\textendash}{\vspace{-5mm}}
		\item{User ist angemeldet}
		\item{Berabeiten/Erstellen-Modus ist geöffnet}
		\item{Ein Eventbild ist ausgewählt.}
	\end{list}
} \\ 
\hline \rule[-2ex]{0pt}{5.5ex} \textbf{Aktion} & {
	\begin{list}{\textendash}{\vspace{-5mm}}
		\item{Der User gibt in das vorgesehene Feld die Bildbeschreibung ein.}
	\end{list}
} \\ 
\hline \rule[-2ex]{0pt}{5.5ex} \textbf{Nachbedingung} & {
	\begin{list}{\textendash}{\vspace{-5mm}}
		\item{Die Bildbeschreibung steht in dem entsprechenden Feld.}
	\end{list}
}  \\
\hline 
\end{tabular}
\newpage
\subsubsection*{Event speichern}
\vspace{2.5mm}
\noindent
\begin{tabular}{|p{0.3\textwidth}||p{0.63\textwidth}|}
\hline \rule[-2ex]{0pt}{5.5ex} \textbf{Vorbedingung} & {
	\begin{list}{\textendash}{\vspace{-5mm}}
		\item{User ist angemeldet}
		\item{Berabeiten/Erstellen-Modus ist geöffnet}
		\item{Mindestens ein Vorstellungstermin ist hinterlegt.}
		\item{Mindestens eine Vorstellungszeit ist hinterlegt.}
		\item{Mindestens eine Preisgruppe ist ausgewählt.}
		\item{Ein Genre ist ausgewählt.}
		\item{Der Name der Veranstaltung wurde angegeben.}
		\item{Die Dauer wurde eingegeben.}
		\item{Eine Beschreibung wurde eingegeben.}
	\end{list}
} \\ 
\hline \rule[-2ex]{0pt}{5.5ex} \textbf{Aktion} & {
	\begin{list}{\textendash}{\vspace{-5mm}}
		\item{Der User klickt auf den Button «Speichern».}
	\end{list}
} \\ 
\hline \rule[-2ex]{0pt}{5.5ex} \textbf{Nachbedingung} & {
	\begin{list}{\textendash}{\vspace{-5mm}}
		\item{Der Event wurde erfasst.}
	\end{list}
}  \\
\hline \rule[-2ex]{0pt}{5.5ex} \textbf{Ausnahme} & {
	\begin{list}{\textendash}{\vspace{-5mm}}
		\item{
			\textbf{Es wurden nicht alle erforderlichen Informationen eingegeben bzw. die eingegebenen Informationen sind nicht valide.}
			
			Der User wird darauf hingewiesen, dass der Event so nicht gespeichert werden kann. Er hat nun die möglichkeit, seine Eingaben zu korrigieren.
		}
	\end{list}
}  \\
\hline 
\end{tabular}


\end{document}
